\documentclass[UTF8]{ctexart}
\usepackage{mathtools,graphicx,abstract,amsmath,booktabs,amsfonts}
\usepackage{pdfpages}
\usepackage{amsmath,amssymb}
\usepackage{float}
\usepackage{longtable}
\usepackage{tabu}
\usepackage{threeparttable}
\usepackage{listings}
\begin{document}
	\title{个人简历}
	\author{程东杰\ 计算机学院\ 人工智能}
	\date{July 05,2022}
	\maketitle
	\setlength{\absrightindent}{0pt}
	\newpage
	\begin{abstract}
	去年我通过了人工智能的转专业选拔,成为人工智能专业的一员。因为有许多专业课需要补,本着打好基础的态度我选择了降级到2021级。经过一年的学习,我掌握了更多的专业知识,积累了实用的经验,对人工智能科研有了更多的向往,也希望能在科研方面有所作为,在提升能力的同时丰富自己的软实力,为未来能有机会进入优秀的国内外学校深造打好基础。下面我将大概介绍自己的基本情况和经历以及科研能力现状。
	\end{abstract}

	{\bf\emph{ 关键字:\ 专业学习能力\  英语能力\ 技术能力\ 我的态度}\rm}
	\newpage
	\tableofcontents
	\newpage
	
	\section{专业学习能力}
	我的专业课成绩一直都比较理想,四门数学已经修完,都在85分以上。截至目前所有必修课程的均分在90分以上。
	\begin{figure}[H]  %h此处,t页顶,b页底,p独立一页,浮动体出现的位置
	\centering  %图表居中
	\includegraphics[width=1\textwidth]{2.png} %图片的名称或者路径之中有空格会出问题 
	\caption{主修中文成绩单 } % 图片标题 
	\label{fig:主修中文成绩单} % 图片标签,可以不写
	\end{figure}
	\begin{figure}[H]  %h此处,t页顶,b页底,p独立一页,浮动体出现的位置
	\centering  %图表居中
	\includegraphics[width=.6\textwidth]{1.png} %图片的名称或者路径之中有空格会出问题 
	\caption{总成绩} % 图片标题 
	\label{fig:主修中文成绩单} % 图片标签,可以不写
	\end{figure}
	\section{英语能力}
	在我看来,做科研的第一步是让自己尽可能跟上"战场"前沿,最有效的方式或许就是大量阅读论文,英语能力自然十分重要。
	\begin{figure}[H]  %h此处,t页顶,b页底,p独立一页,浮动体出现的位置
		\centering  %图表居中
		\includegraphics[width=.7\textwidth]{3.png} %图片的名称或者路径之中有空格会出问题 
		\caption{CET-6成绩单 } % 图片标题 
		\label{fig:主修中文成绩单} % 图片标签,可以不写
	\end{figure}
	在六级考试中,我获得了636这样一个比较理想的分数,我也参加了本学期学校出国留学部参与承办的雅思课程,进一步加强自己全方面的英语能力。
	\section{技术能力}
	\subsection{基础算法}
	这一年里,我学习了传统的算法,在Acwing上刷完了算法基础课,对基础的算法和数据结构的实现有基本的掌握
	\begin{figure}[H]  %h此处,t页顶,b页底,p独立一页,浮动体出现的位置
	\centering  %图表居中
	\includegraphics[width=.6\textwidth]{5.png} %图片的名称或者路径之中有空格会出问题 
	\caption{部分学过的算法} % 图片标题 
	\label{fig:主修中文成绩单} % 图片标签,可以不写
	\end{figure}
	在\textbf{腾讯杯-ACM校赛}中,我和小伙伴进入决赛,但遗憾最后没有取得好的名次。
	\subsection{科研环境}
	\subsubsection{Pytorch框架使用}
	学习了人工智能导论后,使用Pytorch构建CNN同时加入半监督学习实现FashionMnist分类,调包调参或许没什么问题,会基本的使用GPU加速学习,仍在进一步学习相关知识。
	\begin{figure}[H]  %h此处,t页顶,b页底,p独立一页,浮动体出现的位置
		\centering  %图表居中
		\includegraphics[width=1\textwidth]{8.png} %图片的名称或者路径之中有空格会出问题 
		\caption{半监督FashionMnist分类实验报告} % 图片标题 
		\label{fig:主修中文成绩单} % 图片标签,可以不写
	\end{figure}
	\subsubsection{掌握远程服务器训练基本方法}
	对于Linux系统我能较为熟练地进行使用,VSCODE通过ssh进行远程服务器连接,我参与的大创项目中,我在老师指导下,连接老师课题组的服务器进行模型训练。这一大创项目——"基于声纹识别的化工动设备检测",目前获得省级立项。
	\begin{figure}[H]  %h此处,t页顶,b页底,p独立一页,浮动体出现的位置
		\centering  %图表居中
		\includegraphics[width=1\textwidth]{6.png} %图片的名称或者路径之中有空格会出问题 
		\caption{SSH连接服务器} % 图片标题 
		\label{fig:主修中文成绩单} % 图片标签,可以不写
	\end{figure}
	\begin{figure}[H]  %h此处,t页顶,b页底,p独立一页,浮动体出现的位置
	\centering  %图表居中
	\includegraphics[width=1\textwidth]{7.png} %图片的名称或者路径之中有空格会出问题 
	\caption{服务器上模型训练} % 图片标题 
	\label{fig:主修中文成绩单} % 图片标签,可以不写
	\end{figure}
	\subsubsection{掌握Linux基本使用}
	我捣鼓了多种形式的Ubuntu系统,现在有多个Ubuntu环境:移动硬盘Ubuntu系统、WSL子系统、虚拟机
	\subsubsection{其他环境}
	Ubuntu系统中我配置了opencv、openvino等CV研究及部署的基本环境,仍在学习。

	\section{我的态度}
	在我看来,参与到科研或是项目之中是丰富自己羽翼,强化自身技术能力的重要途径之一,我也明白这条路上想要有漂亮的成果很难,但我喜欢未知,喜欢探索,我愿意为探索之路付出我的时间和精力。作为降转的学生,课业的压力会稍微轻一些,意味着我有更多的时间进行科学探索研究,为科学发展贡献一份自己的力量的同时收获经验与技术。
	
\end{document}
